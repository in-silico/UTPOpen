Football, like any other sport, is continually evolving, and there have
been teams which have developed truly marvelous playing styles
throughout the years. Among the most cited examples, you could point to
the national football teams of Brazil, Netherlands and Argentina during
the 70s, as well as to many European clubs that have won glorious titles
in recent decades, such as Dutch club Ajax, Milan of Italy, Barcelona of
Spain and Germany's Bayern München.

One of the most popular styles used in modern football is the so-called
\emph{tiki-taka}, which generally refers to a particular style in which
short passes and movements are critical, and the overarching goal is to
maintain possession of the ball, wearing out the opposing team and
disrupting their defensive lines.

The football manager (FM) of the national team for the country of
Nlogonia---a beautiful place where football is gaining popularity---is
planning a tiki-taka training session with a group of his players. He
knows that, within the framework of this simple style, there is an ample
range of play combinations, and he wants to explore those possibilities.
For this purpose, he starts by giving out instructions to his players,
indicating the possible \emph{passing lines}. A passing line is defined
by two players; one who passes the ball, and one who receives the pass.
The FM expressly forbids passing the ball from a player $A$ to another
player $B$ unless he has defined a passing line from $A$ to $B$.

The FM wants to prepare a special play with $N$ passes, and he chooses
two players, which will be denoted $X$ and $Y$, to be the starting and
ending points of the play---that is, $X$ must make the first pass, and
$Y$ has to receive the last pass. Now, the FM wants to know the number
of different ways in which such play can be executed, performing $N$
passes, starting with $X$ and ending in $Y$. Nlogonia's FM is so busy
these days with the team that he asks your help to create a program that
calculates this value for him.

\subsection{Input}\label{input}

Input starts with a positive integer $T$ ($T \leq 20$), denoting the
number of test cases.

Every test case starts with a blank line. The next line contains two
integers: $P, L$, representing the number of players to be part of the
training session, and the number of passing lines defined by the FM,
respectively.

The next $L$ lines describe the passing lines. Each of these lines
contain two integers: $A, B$, indicating that player $A$ may pass the
ball to player $B$ ($A \neq B$). Take notice that this does not
necessarily imply that $B$ is allowed to pass the ball to $A$. No
passing line will appear more than once in the input.

The next line contains an integer $Q$, the number of queries made by the
FM. The next $Q$ lines describe the queries. Each of these lines contain
three integers: $X, Y, N$, in that order.

~~~ $2 \leq P \leq 11$~~; ~~$1 \leq Q \leq 30$~~;
~~$1 \leq N \leq 10^{15}$~~; ~~$1 \leq A, B, X, Y \leq P$

\subsection{Output}\label{output}

For each test case, your program must print a line with the message
\texttt{Case i:}, replacing \emph{i} with the case number. Then you must
print $Q$ lines, with the answers to the queries, in the same order as
they were presented in the input.

The answer to each query must be the total number of different ways in
which the players can pass the ball $N$ times, starting with $X$ and
ending in $Y$. Since this number can be very large, print this answer
modulo $1000000007$.

\paragraph{}

\vspace{-20pt}

\noindent

\begin{tabular}{|l|l|}
  \hline
  \FFATitle{Sample Input} &
  \FFATitle{Output for Sample Input} \\
  \hline
  \begin{minipage}[t]{0.48\textwidth}
    \vspace{-8pt}
    \begin{verbatim}2

4 6
1 2
2 1
2 3
2 4
3 2
3 4
4
1 4 2
1 4 4
2 4 10
4 1 1

3 4
1 2
2 1
2 3
3 2
1
1 3 30\end{verbatim}
    \vspace{-4pt}
  \end{minipage} &
  \begin{minipage}[t]{0.48\textwidth}
    \vspace{-8pt}
    \begin{verbatim}Case 1:
1
2
16
0
Case 2:
16384\end{verbatim}
    \vspace{-4pt}
  \end{minipage} \\
  \hline
\end{tabular}

\begin{center}\rule{3in}{0.4pt}\end{center}

Problem Setter: Leonardo B.
