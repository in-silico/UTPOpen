\documentclass[12pt]{article}


%%
%% Packages
%%
\usepackage{ctable}
\usepackage{enumerate}
\usepackage{etoolbox}
\usepackage{float}
\usepackage{wrapfig}
\usepackage[letterpaper,top=1cm,bottom=1.5cm,left=1.6cm,right=1.4cm]{geometry}
\usepackage{fixltx2e}
\usepackage{titlesec}
\usepackage{color}
\usepackage{hyperref}
\usepackage{fancyvrb}
\usepackage{microtype}

\usepackage[utf8]{inputenc}
\usepackage[T1]{fontenc}


%%
%% Fonts
%%
\usepackage{eulervm}
\usepackage{libertine}
\usepackage[defaultsans]{opensans}
\usepackage[scaled=0.8]{DejaVuSansMono}

%%
%% Colors
%%
\definecolor{tango-orange-2}{RGB}{245,121,0}
\definecolor{tango-skyblue-1}{RGB}{114,159,207}
\definecolor{tango-skyblue-2}{RGB}{52,101,164}
\definecolor{tango-skyblue-3}{RGB}{32,74,135}
\definecolor{tango-plum-2}{RGB}{117,80,123}


%%
%% Titles and Styles
%%
\newcommand{\FFACommentary}[1]{%
  \vspace{0.4em}\noindent\makebox[\linewidth][c]{%
  \begin{minipage}{0.8\linewidth} \footnotesize #1 \end{minipage}}
  \vspace{-0.2em}}
\newcommand{\FFASpeak}[1]{{\small\slshape #1:\hspace{0.4em}}}
\newcommand{\FFATitle}[1]{{%
  \color{tango-skyblue-2}\sffamily\large\bfseries\hspace{-0.4em} #1}}

\titleformat{\subsection}
{\color{tango-skyblue-2}\sffamily\large\bfseries}
{\color{red}\thesection}{0pt}{}
\titlespacing{\subsection}{0pt}{1em}{0pt}

% Remove the author and date fields and the space associated with them
% from the definition of maketitle!
\makeatletter
\renewcommand{\@maketitle}{
\newpage
 \null
 %\vskip 1em%
 \vspace{-20pt}
 \begin{center}%
  {\color{tango-skyblue-3}\sffamily\Large\bfseries \@title}%
 \end{center}%
 \vspace{-2pt}
} \makeatother


%%
%% Spacing
%%
\renewcommand*\arraystretch{1.5}

\setlength{\parindent}{0pt}
\setlength{\parskip}{6pt plus 2pt minus 1pt}
\setcounter{secnumdepth}{0}


%%
%% Misc.
%%
\hypersetup{
    colorlinks,
    linktocpage,
    citecolor=tango-orange-2,
    filecolor=black,
    linkcolor=tango-plum-2,
    urlcolor=tango-skyblue-2
}
\graphicspath{{build/img/latex/}}


%%
%% Document Body
%%
\title{ Problem ?: World Cup Sticker Album }
\begin{document}
\maketitle

The starting date for the 2014 FIFA World Cup is approaching, and one of
the most popular traditions among followers of this event is to collect
the stickers (which around our latitudes receive many names, such as
\emph{cromos}, \emph{monas}, \emph{caramelos}, \emph{estampas} and
others) for the official album, featuring the players and stadiums that
will be part of the upcoming World Cup.

After weeks of effort, you and some of your friends are close to
completing your albums. You are only a few stickers short of the total
of 639 stickers required to fill the album. And, well, in order to
obtain those few missing stickers, you have two alternatives: one is to
swap stickers with your friends (as tradition dictates), and the other
option---maybe less romantic, but just as effective---is to buy some of
the stickers that you have not been able to get any other way.

Thanks to an arduous investigation, you know your friends' lists of
\emph{requested} stickers (those that they need to complete their
albums), as well as their \emph{offered} stickers (those that they have
one or more duplicates of, hence they are willing to part with them).
Naturally, you also know your own lists of requested and offered
stickers.

Swapping stickers only occurs if you have a sticker that is requested by
one of your friends, and you receive from that same friend one of their
offered stickers. Keep in mind that, although your friends will only
receive stickers they're requesting, you on the other hand realize that
getting a duplicate from a swap may be good in the long run, so you
don't have any restrictions of this kind. Furthermore, you know that you
can do all your trades before your friends get together, so it's safe to
assume that they won't swap stickers among themselves.

As mentioned before, the other way to obtain the stickers you're missing
is to buy them directly. To this end, you have a list of sellers, and
know the stickers they have and their prices. It is possible to swap
stickers you have bought, and you may also assume that your friends
won't buy stickers.

With so many numbers and possible operations going around in your head,
you decide it would be better to write a program that helps you
determine the maximum number of stickers you can obtain from your
missing list, doing it with the least possible cost.

\subsection{Input}\label{input}

Input starts with a positive integer $T$ ($T \leq 300$), which denotes
the number of test cases.

Each case starts with a blank line, followed by the data of your
requested and offered stickers. This information is described in two
lines. The first line starts with an integer $p$, representing the
number of stickers you request, and is followed by a list of $p$
distinct integers: $R_1, R_2, \ldots, R_p$, which are the ID numbers of
the requested stickers. The second line starts with an integer $q$,
representing the number of distinct stickers you offer, followed by a
list of $q$ tuples of integers of the form $(O_i,N_i)$ (with
$1 \leq i \leq q$), where $O_i$ is the ID number of the i-th sticker,
and $N_i$ is the amount of stickers with the number $O_i$ that you
offer.

The next line contains an integer $F$, which indicates the number of
friends available to swap stickers with you. This is followed by $2F$
lines, describing the requested and offered stickers for each of your
friends, using the same format described above (2 lines per person). No
sticker will be listed in both the $R$ and $O$ lists for any given
person---in other words, $R \cap O = \emptyset$ in everyone's data.

The next line contains and integer $S$, denoting the number of sellers.
This is followed by $S$ lines, each line describing the data for a
single seller. These lines start with an integer $k$, representing the
number of distinct stickers for sale, followed by a list of $k$ tuples
of integers of the form $(I_i,M_i,P_i)$, with ($1 \leq i \leq k$), where
$I_i$ is the ID number of the i-th sticker, $M_i$ is the amount of
stickers with the number $I_i$ for sale, and $P_i$ is the price (in
Colombian \emph{pesos}) for each $I_i$ sticker.

The ID numbers of all stickers (in the lists $R, O, I$) will be in the
range $[1:639]$, and the total number of distinct stickers included in
all the $R, O, I$ lists for each test case will never exceed 30.

~~~ $1 \leq F+S < 20$~~; ~~$1 \leq p, q, k \leq 30$~~;
~~$1 \leq N_i, M_i \leq 10$~~; ~~$10^3 \leq P_i \leq 10^4$

\subsection{Output}\label{output}

For each test case, print the case number, followed by two integer
numbers: the number of stickers you will be missing after all your swaps
and purchases, and the total cost for those operations, in that order.
The solution must minimize the number of missing stickers in first
place, and then minimize the cost.

\paragraph{}

\vspace{-20pt}

\noindent

\begin{tabular}{|l|l|}
  \hline
  \FFATitle{Sample Input} &
  \FFATitle{Output for Sample Input} \\
  \hline
  \begin{minipage}[t]{0.48\textwidth}
    \vspace{-8pt}
    \begin{verbatim}2

3 171 202 632
2 (187,1) (264,2)
2
2 187 264
2 (171,1) (632,1)
1 264
1 (202,1)
0

3 500 566 190
1 (125,1)
2
1 125
1 (37,1)
1 37
1 (566,1)
1
2 (500,1,2000) (566,1,1000)\end{verbatim}
    \vspace{-4pt}
  \end{minipage} &
  \begin{minipage}[t]{0.48\textwidth}
    \vspace{-8pt}
    \begin{verbatim}Case 1: 0 0
Case 2: 1 2000\end{verbatim}
    \vspace{-4pt}
  \end{minipage} \\
  \hline
\end{tabular}

\subsection{Explanation of Sample
Cases}\label{explanation-of-sample-cases}

In the first example, you are missing only three stickers for your album
(those stickers are 171, 202 and 632). Luckily for you, two of your
friends are willing to swap the exact stickers you need, and you have
all the stickers they request, so you can complete the album with no
cost at all.

In the second example, you are missing three stickers again, but this
time it won't be possible to fill your album (no one offers you sticker
190). Nevertheless, you can get sticker 566 if you trade with your first
friend 125 for 37, and then swap 37 for 566 with your second friend.
Finally, you can buy sticker 500 for 2000 pesos.

\begin{center}\rule{3in}{0.4pt}\end{center}

Problem Setter: Leonardo B.

\end{document}
