\documentclass[12pt]{article}


%%
%% Packages
%%
\usepackage{ctable}
\usepackage{enumerate}
\usepackage{etoolbox}
\usepackage{float}
\usepackage{wrapfig}
\usepackage[letterpaper,top=1cm,bottom=1.5cm,left=1.6cm,right=1.4cm]{geometry}
\usepackage{fixltx2e}
\usepackage{titlesec}
\usepackage{color}
\usepackage{hyperref}
\usepackage{fancyvrb}
\usepackage{microtype}

\usepackage[utf8]{inputenc}
\usepackage[T1]{fontenc}


%%
%% Fonts
%%
\usepackage{eulervm}
\usepackage{libertine}
\usepackage[defaultsans]{opensans}
\usepackage[scaled=0.8]{DejaVuSansMono}

%%
%% Colors
%%
\definecolor{tango-orange-2}{RGB}{245,121,0}
\definecolor{tango-skyblue-1}{RGB}{114,159,207}
\definecolor{tango-skyblue-2}{RGB}{52,101,164}
\definecolor{tango-skyblue-3}{RGB}{32,74,135}
\definecolor{tango-plum-2}{RGB}{117,80,123}


%%
%% Titles and Styles
%%
\newcommand{\FFACommentary}[1]{%
  \vspace{0.4em}\noindent\makebox[\linewidth][c]{%
  \begin{minipage}{0.8\linewidth} \footnotesize #1 \end{minipage}}
  \vspace{-0.2em}}
\newcommand{\FFASpeak}[1]{{\small\slshape #1:\hspace{0.4em}}}
\newcommand{\FFATitle}[1]{{%
  \color{tango-skyblue-2}\sffamily\large\bfseries\hspace{-0.4em} #1}}

\titleformat{\subsection}
{\color{tango-skyblue-2}\sffamily\large\bfseries}
{\color{red}\thesection}{0pt}{}
\titlespacing{\subsection}{0pt}{1em}{0pt}

% Remove the author and date fields and the space associated with them
% from the definition of maketitle!
\makeatletter
\renewcommand{\@maketitle}{
\newpage
 \null
 %\vskip 1em%
 \vspace{-20pt}
 \begin{center}%
  {\color{tango-skyblue-3}\sffamily\Large\bfseries \@title}%
 \end{center}%
 \vspace{-2pt}
} \makeatother


%%
%% Spacing
%%
\renewcommand*\arraystretch{1.5}

\setlength{\parindent}{0pt}
\setlength{\parskip}{6pt plus 2pt minus 1pt}
\setcounter{secnumdepth}{0}


%%
%% Misc.
%%
\hypersetup{
    colorlinks,
    linktocpage,
    citecolor=tango-orange-2,
    filecolor=black,
    linkcolor=tango-plum-2,
    urlcolor=tango-skyblue-2
}
\graphicspath{{build/img/latex/}}


%%
%% Document Body
%%
\title{ Problem ?: Square Formations }
\begin{document}
\maketitle

Raúl and Jaime, as you may recall, are two young and bright brothers who
love games. Jaime has received as a present a set of little plastic
soldiers, and he happily shares it with Raúl every time they get
together to play.

Their greatest source of amusement comes from arranging their groups of
soldiers in different ways. Raúl, for instance, likes to vary his
formation patterns and arrange his platoons in circular or elliptic
shapes, while Jaime undoubtedly prefers square formations for his
soldiers---in which the number of rows is equal to the number of
columns.

One day, Raúl organizes a surprise attack using two platoons of
soldiers, attacking two opposing flanks. This baffles Jaime, who after a
long pause to think about it, discovers that when he divides his square
formation into two, it becomes impossible to reassemble the halves in
new square formations, and this happens regardless of the number of
soldiers in his original formation.

Raúl pauses to think about this too, and eventually presents Jaime with
an alternative: he can prepare a first group of $N$ soldiers ($N$ being
a positive, even integer), and maintain two auxiliary groups of $K$
soldiers. If $N$ and $K$ are chosen carefully, Raúl explains, then it is
possible to create square formations with $N+K$ and $(N/2)+K$ soldiers.

Jaime likes the elegance of this idea, and doesn't take long to discover
an example that illustrates Raúl's suggestion: if he keeps a main group
of 14 soldiers, and two auxiliary groups with 2 soldiers each, then he
can use one of the auxiliary groups to produce a main formation of 16
soldiers, and if he needs to split his main platoon into two, then he
can produce two minor square formations of 9 soldiers. So, $N=14, K=2$
is a valid choice for Jaime.

According to this, we'll say that a pair of integers $(N, K)$ is valid
if it holds for Raúl's proposition; that is, if it is possible to
produce square formations with groups of $N+K$ and $(N/2)+K$ soldiers.
The question in their minds now is: given an arbitrary value of $K$,
what would be the lowest possible $N$ that generates a valid pair?

\subsection{Input}\label{input}

Input starts with a positive integer $T$, that denotes the number of
test cases.

Each test case is described by a single integer $K$ in its own line. It
can be safely assumed that all test cases have a solution in which $N$
is a number not greater than $10^6$.

~~~ $T \leq 1000$~~; ~~$1 \leq K \leq 1000$

\subsection{Output}\label{output}

For each test case, print the case number, followed by the value of $N$.
This number must be the lowest integer possible, such that $(N, K)$
forms a valid pair according to Raúl's proposition.

\paragraph{}

\vspace{-20pt}

\noindent

\begin{tabular}{|l|l|}
  \hline
  \FFATitle{Sample Input} &
  \FFATitle{Output for Sample Input} \\
  \hline
  \begin{minipage}[t]{0.48\textwidth}
    \vspace{-8pt}
    \begin{verbatim}3
1
2
36\end{verbatim}
    \vspace{-4pt}
  \end{minipage} &
  \begin{minipage}[t]{0.48\textwidth}
    \vspace{-8pt}
    \begin{verbatim}Case 1: 48
Case 2: 14
Case 3: 1728\end{verbatim}
    \vspace{-4pt}
  \end{minipage} \\
  \hline
\end{tabular}

\begin{center}\rule{3in}{0.4pt}\end{center}

Problem Setter: Leonardo B.

\end{document}
